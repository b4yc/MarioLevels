%%%%%%%%%%%%%%%%%%%%%%%%%%%%%%%%%%%%%%%%%%%%%%%%%%%%%%%%%%%%%%%%%%%%%%%%%%%%%%
% Documento de ejemplo para la plantilla fiudcPFC.cls (version: 2010/05/12)
%%%%%%%%%%%%%%%%%%%%%%%%%%%%%%%%%%%%%%%%%%%%%%%%%%%%%%%%%%%%%%%%%%%%%%%%%%%%%%
\documentclass[twoside,english]{fiudcPFC} % El idioma se puede definir como castellano, inglés o gallego
\usepackage{makeidx}            % Este paquete permite crear glosarios de términos (índice alfabético).
                                % Se puede eliminar sino se emplea este tipo de índice.
\usepackage{subfigure}          % Permite incluir figuras con múltiples subfiguras dentro (opcional, se puede eliminar).
\usepackage{longtable}          % Permite usar tablas que ocupen más de una hoja (opcional, se puede eliminar).
\usepackage{multirow}           % Permite agrupar filas en las tablas (opcional, se puede eliminar).
\usepackage{algpseudocode}		% Soporte de algoritmos botones
\usepackage[chapter]{algorithm} %   "           "

\usepackage{hyperref}			% Puede ser aconsejable establecer el driver, como abajo
%\usepackage[dvipdfm]{hyperref}  % Hyperenlaces en el documento pdf generado.
                                % Es aconsejable que sea el último paquete a incluir ya que redefine comandos de otros.


\dirFiguras{./figures/}     % Define el subdirectorio que contiene las figuras de la memoria.
                                % Es muy recomendable que sea un directorio relativo como el del ejemplo.
                                % Si no se indica ninguno, o se elimina, toma por defecto el directorio actual.
\dirCapitulos{./chapters/}     % Define el subdirectorio que contiene los capítulos de la memoria (*.tex).
                                % Las recomendaciones anteriores se aplican también en este caso.

\title{Este será el título del proyecto} % Título del proyecto
\department{Computation}        % Departamento.
\degree{i}                      % Grado (i: ingeniería, g: gestión, s: sistemas, t: tesis doctoral).
\author[m]{Jorge Diz Pico}% El parámetro opcional indica el sexo: f para femenino y m para masculino.
\tutor[m]{David Camacho}   % Admite también parámetro opcional de sexo.
\director[f]{Bertha Guijarro Berdiñas}% Admite también parámetro opcional de sexo.

\pfcdate{June 2012}         % Fecha del proyecto.
\dedication{To life and death}       % Dedicatoria.

\makeindex % Comando para indicar que se creará un glosario de términos.


\begin{document}

\maketitle

\frontmatter

\incluir{acknowledgements}

\incluir{summary}

%%%%%%%%%%%%%%%%%%%%%%%%%%%%
% INDICES
%%%%%%%%%%%%%%%%%%%%%%%%%%%%
\indices{0em} % Este parámetro indica la separación entre párrafos (títulos) para los indices
{
 % Se puede comentar alguno de los siguientes comandos si no se desea alguno de los índices asociados
 \listoffigures %
 \listoftables %
 \listofalgorithms %
}

\mainmatter

%%%%%%%%%%%%%%%%%%%%%%%%%%%%
% CAPÍTULOS DE LA MEMORIA
%%%%%%%%%%%%%%%%%%%%%%%%%%%%
% Se incluye cada capítulo en un fichero .tex independiente.
% El directorio es el indicado por el comando \dirCapitulos definido al comienzo de este documento.
\incluir{sections}
\incluir{figures}
\incluir{tables}
\incluir{algorithms}
\incluir{equations}
\incluir{glossary}

\appendix

\incluir{appendix1}
\incluir{appendix2}

\backmatter

%%%%%%%%%%%%%%%%%%%%%%%%%%%%%%%%%
% BIBLIOGRAFÍA
%%%%%%%%%%%%%%%%%%%%%%%%%%%%%%%%%
\bibliographystyle{./bibliography/aplain}
\addcontentsline{toc}{chapter}{Bibliography}
\bibliography{./bibliography/example}

\printindex % Para mostrar glosario de términos

\end{document}
