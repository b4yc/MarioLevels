%%%%%%%%%%%%%%%%%%%%%%%%%%%%%%%%%%%%%%%%%%%%%%%%%%%%%%%%%%%%%%%%%%%%%%%%
\chapter{Manejo de ecuaciones}
%%%%%%%%%%%%%%%%%%%%%%%%%%%%%%%%%%%%%%%%%%%%%%%%%%%%%%%%%%%%%%%%%%%%%%%%

En este apartado se muestran algunos ejemplos de uso de ecuaciones en el texto.

%%%%%%%%%%%%%%%%%%%%%%%%%%%%%%%%%%%%%%%%%%%%%%%%%%%%%%%%%%%%%%%%%%%%%%%%
\section{Ejemplo de creaciín y citas de ecuaciones.}
%%%%%%%%%%%%%%%%%%%%%%%%%%%%%%%%%%%%%%%%%%%%%%%%%%%%%%%%%%%%%%%%%%%%%%%%

Ahora definimos la siguiente ecuaciín:
\begin{equation}
S_i^{(m)} = \lim \limits_{\sigma \rightarrow 0
}\frac{\sqrt{var(\Delta y_i^{(m)})}}{\sigma} \ ,
\label{ecuacion1}
\end{equation}
A continuaciín se referencia la ecuaciín (\ref{ecuacion1}).

Otro ejemplo de ecuaciín:
\begin{equation}
\int_{2}^{5} f(x) dx
\label{ecuacion2}
\end{equation}
y su referencia correspondiente: ecuaciín (\ref{ecuacion2}).

Un par de ejemplos mís de ecuaciones de míltiples filas pero con una ínica numeraciín. Primer ejemplo:
%
\begin{equation}
\begin{array}{c}
g_i^{(m)}(h_{i1}^{(m)}(y_{1}^{(m-1)}),\dots,h_{iN_{m-1}}^{(m)}(y_{N_{m-1}}^{(m-1 )})) =
\\[0.3cm]%
\tanh (h_{i1}^{(m)}(y_{1}^{(m-1)}) + \dots + h_{iN_{m-1}}^{(m)}(y_{N_{m-1}}^{(m-1)})) \ ,
\\[0.3cm]%
\phi_{ijz}^{(m)}(y_j^{(m-1)}) = y_j^{(m-1)},
\\[0.3cm]%
n_{ij}^{(m)} = 1; \ \forall i,j.
\end{array}
\label{ecuacion3}
\end{equation}
%
Segundo ejemplo:
%
\begin{equation}
\begin{array}{rcl}
y & = & {\displaystyle \frac{df}{dz}} \\
  & = & (x + 2)^2                     \\
  & = & x^2 + 2x + 4                  \\
  & = & 3
\end{array}
\label{ecuacion4}
\end{equation}
%
Ahora referencio las ecuaciones (\ref{ecuacion3}) y (\ref{ecuacion4}).

Ejemplo de otra ecuaciín de míltiples filas sin numerar y alineadas en relaciín al primer símbolo $=$ de cada fila:
%
\begin{eqnarray*}
E[\Delta y_i^{(1)}] &=& E\left[\sum
\limits_{r=1}^{N_{0}} \left( \frac{\partial y_i^{(1)}}{\partial
y_r^{(0)}} \Delta y_r^{(0)} + \sum \limits_{s=1}^{n_{ir}^{(1)}}
\frac{\partial y_i^{(1)}}{\partial a_{irs}^{(1)}} \Delta
a_{irs}^{(1)} \right)\right] \\
&=& \sum \limits_{r=1}^{N_{0}} \left( \frac{\partial y_i^{(1)}}{\partial
y_r^{(0)}} E[\Delta y_r^{(0)}] + \sum \limits_{s=1}^{n_{ir}^{(1)}}
\frac{\partial y_i^{(1)}}{\partial a_{irs}^{(1)}} E[\Delta
a_{irs}^{(1)}] \right) = 0
\end{eqnarray*}
%
Un íltimo ejemplo en el que se usa una expresiín matemítica que se engloba con una llave.
%
\begin{equation}
\left\{ \begin{array}{ll}
  y = x^2, z = x & \mbox{if $x > 0$} \\
  y = 0, z = 0   & \mbox{if $x \leq 0$}
\end{array} \right.
\end{equation}
